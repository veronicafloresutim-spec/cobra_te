\documentclass[12pt,letterpaper]{article}
\usepackage[utf8]{inputenc}
\usepackage[spanish]{babel}
\usepackage{geometry}
\usepackage{fancyhdr}
\usepackage{graphicx}
\usepackage{longtable}
\usepackage{array}
\usepackage{xcolor}
\usepackage{amsmath}
\usepackage{hyperref}
\usepackage{booktabs}
\usepackage{multirow}
\usepackage{colortbl}

% Configuración de página
\geometry{top=2.5cm, bottom=2.5cm, left=2.5cm, right=2.5cm}
\setlength{\headheight}{30pt}

% Definición de colores corporativos
\definecolor{cobrablue}{RGB}{103,80,164}
\definecolor{cobraorange}{RGB}{255,107,53}
\definecolor{cobragreen}{RGB}{76,175,80}
\definecolor{cobragray}{RGB}{73,69,79}

% Configuración de encabezados y pies de página
\pagestyle{fancy}
\fancyhf{}
\fancyhead[L]{\includegraphics[height=20pt]{Logo.png}}
\fancyhead[C]{\textcolor{cobrablue}{\textbf{COBRA TE - Sistema POS Integral}}}
\fancyhead[R]{\textcolor{cobragray}{\textbf{Acta de Alcance}}}
\fancyfoot[C]{\textcolor{cobragray}{Página \thepage}}
\fancyfoot[R]{\textcolor{cobragray}{\today}}

% Configuración de enlaces
\hypersetup{
    colorlinks=true,
    linkcolor=cobrablue,
    filecolor=cobrablue,
    urlcolor=cobrablue,
}

\begin{document}

% Portada
\begin{titlepage}
    \centering
    \vspace*{1cm}
    
    % Logotipo principal
    \includegraphics[width=4cm]{Logo.png}
    
    \vspace{1cm}
    {\color{cobrablue}\Huge\textbf{COBRA TE}}
    
    \vspace{0.5cm}
    {\color{cobraorange}\Large Sistema de Punto de Venta Integral}
    
    \vspace{2cm}
    {\color{cobragray}\LARGE\textbf{ACTA DE ALCANCE}}
    
    \vspace{1cm}
    {\color{cobragray}\large Versión 1.6}
    
    \vspace{2cm}
    \begin{tabular}{|p{4cm}|p{8cm}|}
        \hline
        \rowcolor{cobrablue!20}
        \textbf{Proyecto} & Sistema POS Cobra Te - Plataforma Multiplataforma \\
        \hline
        \textbf{Cliente} & Establecimiento Cobra Te \\
        \hline
        \textbf{Responsable} & Verónica Flores \\
        \hline
        \textbf{Fecha} & 3 de octubre de 2025 \\
        \hline
        \textbf{Estado} & Aprobado \\
        \hline
    \end{tabular}
    
    \vfill
    {\color{cobragray}\large Documento Confidencial}
    
\end{titlepage}

\newpage

% Índice
\tableofcontents
\newpage

% Histórico de Cambios
\section*{Histórico de Cambios}
\addcontentsline{toc}{section}{Histórico de Cambios}

\begin{longtable}{|p{3cm}|p{2cm}|p{4cm}|p{6cm}|}
\hline
\rowcolor{cobrablue!20}
\textbf{Fecha de Revisión} & \textbf{Versión} & \textbf{Responsable(s)} & \textbf{Cambios} \\
\hline
\endhead
3 de octubre 2025 & 1.0 & Verónica Flores & Creación inicial del documento de alcance del proyecto \\
\hline
3 de octubre 2025 & 1.1 & Verónica Flores & Actualización de tecnologías: React.js para web, React Native para móvil \\
\hline
3 de octubre 2025 & 1.2 & Verónica Flores & Actualización de firmas de aprobación del documento \\
\hline
3 de octubre 2025 & 1.3 & Verónica Flores & Reestructuración de entregables en formato de tabla detallada \\
\hline
3 de octubre 2025 & 1.4 & Verónica Flores & Actualización de layout con logotipo oficial Logo.png \\
\hline
3 de octubre 2025 & 1.5 & Verónica Flores & Mejoras de layout: firmas centradas y formato de tablas optimizado \\
\hline
3 de octubre 2025 & 1.6 & Verónica Flores & Eliminación de fechas, criterios de éxito y encabezados de fases \\
\hline
\end{longtable}

\newpage

% 1. Introducción
\section{Introducción}

\subsection{Propósito del Documento}
Este documento establece el alcance del proyecto ``Cobra Te - Sistema POS Integral'', definiendo los objetivos, entregables y limitaciones para el desarrollo de una solución tecnológica completa que abarca desde una aplicación de escritorio hasta implementaciones web y móviles.

\subsection{Antecedentes}
El establecimiento Cobra Te requiere modernizar sus procesos de venta mediante la implementación de un sistema de punto de venta (POS) que permita:
\begin{itemize}
    \item Gestión eficiente de transacciones de venta
    \item Control de inventario en tiempo real
    \item Administración de usuarios y roles
    \item Generación de reportes y análisis de ventas
    \item Escalabilidad hacia plataformas web y móviles
\end{itemize}

\section{Descripción del Proyecto}

\subsection{Visión General}
Cobra Te - Sistema POS Integral es una solución tecnológica multiplataforma diseñada para optimizar las operaciones comerciales del establecimiento, proporcionando herramientas robustas para la gestión de ventas, inventario y administración, con capacidad de expansión hacia el comercio electrónico y gestión móvil.

\subsection{Objetivos del Proyecto}

\subsubsection{Objetivo General}
Desarrollar e implementar un sistema integral de punto de venta que automatice y optimice los procesos comerciales de Cobra Te, con arquitectura escalable para futuras expansiones web y móviles.

\subsubsection{Objetivos Específicos}
\begin{enumerate}
    \item Crear una aplicación de escritorio JavaFX para operaciones POS principales
    \item Implementar un sistema de gestión de usuarios con roles diferenciados
    \item Desarrollar módulos de gestión de productos y categorías
    \item Establecer un sistema de reportes y análisis de ventas
    \item Diseñar una arquitectura backend preparada para expansión web
    \item Planificar la implementación de interfaces web responsivas
    \item Conceptualizar aplicaciones móviles para Android e iOS
\end{enumerate}

\section{Alcance del Proyecto}

\subsection{Fase 1: Aplicación de Escritorio (Actual)}

\subsubsection{Módulos Incluidos}
\begin{itemize}
    \item \textbf{Sistema de Autenticación}
    \begin{itemize}
        \item Login seguro con encriptación SHA-256
        \item Gestión de sesiones de usuario
        \item Recuperación de contraseñas
    \end{itemize}
    
    \item \textbf{Gestión de Usuarios}
    \begin{itemize}
        \item Registro de cajeros y administradores
        \item Asignación de roles y permisos
        \item Gestión de información personal
    \end{itemize}
    
    \item \textbf{Sistema de Punto de Venta}
    \begin{itemize}
        \item Interfaz de venta intuitiva
        \item Cálculo automático de totales e impuestos
        \item Múltiples métodos de pago
        \item Generación de tickets de venta
    \end{itemize}
    
    \item \textbf{Gestión de Inventario}
    \begin{itemize}
        \item Administración de productos
        \item Gestión de categorías
        \item Control de stock en tiempo real
        \item Alertas de inventario bajo
    \end{itemize}
    
    \item \textbf{Reportes y Análisis}
    \begin{itemize}
        \item Reportes de ventas diarias, semanales y mensuales
        \item Análisis de productos más vendidos
        \item Estadísticas de desempeño por cajero
    \end{itemize}
\end{itemize}

\subsection{Fase 2: Expansión Web (Planificada)}

\subsubsection{Componentes Web}
\begin{itemize}
    \item \textbf{Portal de Administración Web}
    \begin{itemize}
        \item Dashboard ejecutivo con métricas clave
        \item Gestión remota de usuarios y productos
        \item Configuración de parámetros del sistema
        \item Backup y restauración de datos
    \end{itemize}
    
    \item \textbf{E-commerce Integration}
    \begin{itemize}
        \item Catálogo de productos en línea
        \item Carrito de compras y checkout
        \item Integración con sistemas de pago digitales
        \item Gestión de pedidos en línea
    \end{itemize}
    
    \item \textbf{API RESTful}
    \begin{itemize}
        \item Servicios web para integración de terceros
        \item Sincronización con sistemas externos
        \item Webhooks para notificaciones en tiempo real
    \end{itemize}
\end{itemize}

\subsection{Fase 3: Aplicaciones Móviles (Futuro)}

\subsubsection{App Móvil para Gestores}
\begin{itemize}
    \item Monitoreo de ventas en tiempo real
    \item Notificaciones push de alertas importantes
    \item Gestión básica de inventario
    \item Reportes móviles con gráficas interactivas
\end{itemize}

\subsubsection{App Cliente}
\begin{itemize}
    \item Programa de lealtad y puntos
    \item Pedidos anticipados y reservas
    \item Notificaciones de ofertas especiales
    \item Historial de compras personalizado
\end{itemize}

\section{Entregables del Producto}

\begin{longtable}{|>{\centering}p{1cm}|p{6cm}|p{8cm}|>{\centering\arraybackslash}p{3cm}|}
\hline
\rowcolor{cobrablue!20}
\textbf{No} & \textbf{Nombre del entregable} & \textbf{Descripción del entregable} & \textbf{Medio (electrónico/impreso/web)} \\
\hline
\endhead

1 & Aplicación POS Desktop & Sistema JavaFX completamente funcional con módulos de venta, inventario, usuarios y reportes & Electrónico / Instalador \\
\hline

2 & Base de Datos & MariaDB configurada con esquemas, datos de prueba y procedimientos almacenados & Electrónico / Scripts SQL \\
\hline

3 & Manual de Usuario & Documentación completa para operadores del sistema POS con capturas de pantalla y flujos & Electrónico / PDF \\
\hline

4 & Documentación Técnica & Arquitectura del sistema, diagramas UML, especificaciones de API y base de datos & Electrónico / PDF \\
\hline

5 & Plan de Pruebas & Casos de prueba, scripts de testing y reportes de validación del sistema & Electrónico / PDF \\
\hline

6 & Scripts de Instalación & Archivos de configuración automática, instaladores y guías de despliegue & Electrónico / Ejecutables \\
\hline

7 & Portal Web React & Aplicación web responsive con dashboard administrativo y gestión remota & Web / Aplicación en línea \\
\hline

8 & API REST & Servicios web documentados con endpoints para integración y sincronización & Electrónico / Documentación \\
\hline

9 & Documentación API & Especificaciones técnicas, ejemplos de uso y guías de integración para desarrolladores & Web / Portal de documentación \\
\hline

10 & E-commerce Integration & Módulo de ventas en línea integrado con inventario físico & Web / Plataforma en línea \\
\hline

11 & Certificados SSL & Configuración de seguridad, certificados digitales y protocolos de encriptación & Electrónico / Configuración \\
\hline

12 & App React Native & Aplicación móvil multiplataforma para gestión y monitoreo & Electrónico / Tiendas de apps \\
\hline

13 & Sistema de Notificaciones & Servicio de push notifications integrado con alertas del sistema & Electrónico / Servicio cloud \\
\hline

14 & Analytics Dashboard & Panel de métricas móviles con reportes interactivos y KPIs en tiempo real & Web / Aplicación móvil \\
\hline

15 & Guías de Publicación & Documentación para publicar en App Store y Google Play con procesos de aprobación & Electrónico / PDF \\
\hline

16 & App Cliente (Opcional) & Aplicación para clientes finales con programa de lealtad y pedidos & Electrónico / Tiendas de apps \\
\hline
\end{longtable}

\section{Arquitectura Tecnológica}

\subsection{Tecnologías Actuales}
\begin{itemize}
    \item \textbf{Frontend Desktop}: JavaFX 21.0.6 con Material Design 3
    \item \textbf{Backend}: Java con arquitectura MVC
    \item \textbf{Base de Datos}: MariaDB 12.0.2
    \item \textbf{Seguridad}: Encriptación SHA-256 para contraseñas
    \item \textbf{Gestión de Dependencias}: Maven
\end{itemize}

\subsection{Tecnologías Planificadas}
\begin{itemize}
    \item \textbf{Frontend Web}: React.js con arquitectura SPA
    \item \textbf{Backend API}: Node.js o extensión del backend Java existente
    \item \textbf{Móvil}: React Native (multiplataforma)
    \item \textbf{Base de Datos}: MariaDB (consistencia con versión actual)
    \item \textbf{Infraestructura Cloud}: Por definir según necesidades del proyecto
    \item \textbf{Containerización}: Docker para desarrollo y despliegue
\end{itemize}

\section{Restricciones y Limitaciones}

\subsection{Restricciones Técnicas}
\begin{itemize}
    \item La aplicación de escritorio debe funcionar en Windows 10+
    \item Requerimiento mínimo de Java 17 o superior
    \item Base de datos debe soportar transacciones ACID
    \item Interfaz debe ser accesible y seguir estándares de usabilidad
\end{itemize}

\subsection{Restricciones de Negocio}
\begin{itemize}
    \item Presupuesto limitado para infraestructura cloud (Fase 2-3)
    \item Tiempo de desarrollo escalonado por fases
    \item Necesidad de mantener operaciones durante implementación
    \item Capacitación del personal debe ser mínima
\end{itemize}

\subsection{Limitaciones Actuales}
\begin{itemize}
    \item Funcionalidad offline limitada (solo aplicación de escritorio)
    \item Integración con sistemas externos pendiente
    \item Escalabilidad horizontal no implementada
    \item Análisis avanzado de datos en desarrollo futuro
\end{itemize}

\section{Gestión de Riesgos}

\subsection{Riesgos Técnicos}
\begin{itemize}
    \item \textbf{Corrupción de datos}: Implementación de backups automáticos diarios
    \item \textbf{Fallos de conectividad}: Modo offline con sincronización posterior
    \item \textbf{Incompatibilidad de versiones}: Versionado estricto y testing continuo
\end{itemize}

\subsection{Riesgos de Negocio}
\begin{itemize}
    \item \textbf{Resistencia al cambio}: Plan de capacitación gradual y soporte continuo
    \item \textbf{Sobrecarga operativa}: Implementación por fases con periodos de adaptación
    \item \textbf{Escalamiento de costos}: Arquitectura modular con crecimiento controlado
\end{itemize}

\section{Cronograma de Alto Nivel}

\begin{longtable}{|p{3cm}|p{12cm}|}
\hline
\rowcolor{cobrablue!20}
\textbf{Fase} & \textbf{Entregables Principales} \\
\hline
\endhead
Fase 1 & Aplicación de escritorio completa, base de datos, documentación \\
\hline
Fase 2 & Portal web React.js, API REST, integración e-commerce \\
\hline
Fase 3 & App React Native multiplataforma, notificaciones, analytics \\
\hline
\end{longtable}

\section{Conclusiones}

El proyecto Cobra Te - Sistema POS Integral representa una solución tecnológica escalable y robusta que evolucionará desde una aplicación de escritorio hasta un ecosistema digital completo. La arquitectura propuesta permite un crecimiento orgánico y controlado, asegurando que cada fase agregue valor tangible al negocio mientras prepara la base para futuras expansiones.

La implementación por fases garantiza la continuidad operativa y permite la adaptación gradual del personal, minimizando riesgos mientras maximiza el retorno de inversión en cada etapa del proyecto.

\vspace{2cm}

\begin{center}
\textbf{Aprobaciones}

\vspace{1.5cm}

\begin{tabular}{>{\centering}p{6cm}>{\centering\arraybackslash}p{6cm}}
\rule{5cm}{0.4pt} & \rule{5cm}{0.4pt} \\[0.5cm]
Mtra. Patricia Mendoza Crisóstomo & Fecha \\
Supervisora Académica & \\[1cm]
\rule{5cm}{0.4pt} & \rule{5cm}{0.4pt} \\[0.5cm]
Líder del Proyecto & Fecha \\
& \\
\end{tabular}
\end{center}

\end{document}